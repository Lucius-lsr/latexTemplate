% !Mode:: "TeX:UTF-8"
% !TEX program  = xelatex

\documentclass{cumcmthesis}
%\documentclass[withoutpreface,bwprint]{cumcmthesis} %去掉封面与编号页

\title{全国大学生数学建模竞赛编写的 \LaTeX{} 模板}
\tihao{A}
\baominghao{4321}
\schoolname{清华大学}
\membera{李胜锐}
\memberb{丁琪龙}
\memberc{刘长昊}
\supervisor{老师}
\yearinput{2017}
\monthinput{08}
\dayinput{22}

\begin{document}

 \maketitle
 \begin{abstract}
这里是摘要
\end{abstract}

%目录
\tableofcontents

\newpage

\section{问题重述}
\subsection{问题背景}
\subsection{提出问题}



\section{问题分析}

\section{假设与符号}

\section{模型建立与求解}

\section{模型的检验}

\section{进一步讨论}

\section{模型的优缺点}

\newpage

%参考文献
\begin{thebibliography}{9}%宽度9
    \bibitem[1]{liuhaiyang2013latex}
    刘海洋.
    \newblock \LaTeX {}入门\allowbreak[J].
    \newblock 电子工业出版社, 北京, 2013.
    \bibitem[2]{mathematical-modeling}
    全国大学生数学建模竞赛论文格式规范 (2013 年 8 月 26 日修改).
\end{thebibliography}

\newpage
%附录
\begin{appendices}

\section{模板所用的宏包}
\begin{table}[htbp]
    \centering
    \caption{宏包罗列}
    \begin{tabular}{ccccc}
        \toprule
        \multicolumn{5}{c}{模板中已经加载的宏包} \\
        \midrule
        amsbsy & amsfonts & {amsgen} & {amsmath} & {amsopn} \\
        amssymb & amstext & {appendix} & {array} & {atbegshi} \\
        atveryend & auxhook & {bigdelim} & {bigintcalc} & {bigstrut} \\
        bitset & bm    & {booktabs} & {calc} & {caption} \\
        caption3 & CJKfntef & {cprotect} & {ctex} & {ctexhook} \\
        ctexpatch & enumitem & {etexcmds} & {etoolbox} & {everysel} \\
        expl3 & fix-cm & {fontenc} & {fontspec} & {fontspec-xetex} \\
        geometry & gettitlestring & {graphics} & {graphicx} & {hobsub} \\
        hobsub-generic & hobsub-hyperref & {hopatch} & {hxetex} & {hycolor} \\
        hyperref & ifluatex & {ifpdf} & {ifthen} & {ifvtex} \\
        ifxetex & indentfirst & {infwarerr} & {intcalc} & {keyval} \\
        kvdefinekeys & kvoptions & {kvsetkeys} & {l3keys2e} & {letltxmacro} \\
        listings & longtable & {lstmisc} & {ltcaption} & {ltxcmds} \\
        multirow & nameref & {pdfescape} & {pdftexcmds} & {refcount} \\
        rerunfilecheck & stringenc & {suffix} & {titletoc} & {tocloft} \\
        trig  & ulem  & {uniquecounter} & {url} & {xcolor} \\
        xcolor-patch & xeCJK & {xeCJKfntef} & {xeCJK-listings} & {xparse} \\
        xtemplate & zhnumber &       &       &  \\
        \bottomrule
    \end{tabular}%
    \label{tab:addlabel}%
\end{table}%

以上宏包都已经加载过了,不要重复加载它们。

\section{排队算法--matlab 源程序}

\begin{lstlisting}[language=matlab]
kk=2;[mdd,ndd]=size(dd);
while ~isempty(V)
[tmpd,j]=min(W(i,V));tmpj=V(j);
for k=2:ndd
[tmp1,jj]=min(dd(1,k)+W(dd(2,k),V));
tmp2=V(jj);tt(k-1,:)=[tmp1,tmp2,jj];
end
tmp=[tmpd,tmpj,j;tt];[tmp3,tmp4]=min(tmp(:,1));
if tmp3==tmpd, ss(1:2,kk)=[i;tmp(tmp4,2)];
else,tmp5=find(ss(:,tmp4)~=0);tmp6=length(tmp5);
if dd(2,tmp4)==ss(tmp6,tmp4)
ss(1:tmp6+1,kk)=[ss(tmp5,tmp4);tmp(tmp4,2)];
else, ss(1:3,kk)=[i;dd(2,tmp4);tmp(tmp4,2)];
end;end
dd=[dd,[tmp3;tmp(tmp4,2)]];V(tmp(tmp4,3))=[];
[mdd,ndd]=size(dd);kk=kk+1;
end; S=ss; D=dd(1,:);
 \end{lstlisting}

 \section{规划解决程序--lingo源代码}

\begin{lstlisting}[language=c]
kk=2;
[mdd,ndd]=size(dd);
while ~isempty(V)
    [tmpd,j]=min(W(i,V));tmpj=V(j);
for k=2:ndd
    [tmp1,jj]=min(dd(1,k)+W(dd(2,k),V));
    tmp2=V(jj);tt(k-1,:)=[tmp1,tmp2,jj];
end
    tmp=[tmpd,tmpj,j;tt];[tmp3,tmp4]=min(tmp(:,1));
if tmp3==tmpd, ss(1:2,kk)=[i;tmp(tmp4,2)];
else,tmp5=find(ss(:,tmp4)~=0);tmp6=length(tmp5);
if dd(2,tmp4)==ss(tmp6,tmp4)
    ss(1:tmp6+1,kk)=[ss(tmp5,tmp4);tmp(tmp4,2)];
else, ss(1:3,kk)=[i;dd(2,tmp4);tmp(tmp4,2)];
end;
end
    dd=[dd,[tmp3;tmp(tmp4,2)]];V(tmp(tmp4,3))=[];
    [mdd,ndd]=size(dd);
    kk=kk+1;
end;
S=ss;
D=dd(1,:);
 \end{lstlisting}
\end{appendices}

\end{document}
