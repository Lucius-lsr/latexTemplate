
\documentclass{article}

\usepackage[version=3]{mhchem} % Package for chemical equation typesetting
\usepackage{siunitx} % Provides the \SI{}{} and \si{} command for typesetting SI units
\usepackage{graphicx} % Required for the inclusion of images
\usepackage{natbib} % Required to change bibliography style to APA
\usepackage{amsmath} % Required for some math elements 
\usepackage{listings} % Required for codes
\usepackage{color}
\definecolor{dkgreen}{rgb}{0,0.6,0}
\definecolor{gray}{rgb}{0.5,0.5,0.5}
\definecolor{mauve}{rgb}{0.58,0,0.82}
\lstset{ %
language={[x86masm]Assembler},                % the language of the code
basicstyle=\footnotesize,           % the size of the fonts that are used for the code
numbers=left,                   % where to put the line-numbers
numberstyle=\tiny\color{gray},  % the style that is used for the line-numbers
stepnumber=2,                   % the step between two line-numbers. If it's 1, each line 
                                % will be numbered
numbersep=5pt,                  % how far the line-numbers are from the code
backgroundcolor=\color{white},      % choose the background color. You must add \usepackage{color}
showspaces=false,               % show spaces adding particular underscores
showstringspaces=false,         % underline spaces within strings
showtabs=false,                 % show tabs within strings adding particular underscores
frame=single,                   % adds a frame around the code
rulecolor=\color{black},        % if not set, the frame-color may be changed on line-breaks within not-black text (e.g. commens (green here))
tabsize=2,                      % sets default tabsize to 2 spaces
captionpos=b,                   % sets the caption-position to bottom
breaklines=true,                % sets automatic line breaking
breakatwhitespace=false,        % sets if automatic breaks should only happen at whitespace
title=\lstname,                 % show the filename of files included with \lstinputlisting;
                                % also try caption instead of title
keywordstyle=\color{blue},          % keyword style
commentstyle=\color{dkgreen},       % comment style
stringstyle=\color{mauve},         % string literal style
escapeinside={\%*}{*)},            % if you want to add LaTeX within your code
morekeywords={*,...}               % if you want to add more keywords to the set
\setlength\parindent{0pt} % Removes all indentation from paragraphs
}

\renewcommand{\labelenumi}{\alph{enumi}.} % Make numbering in the enumerate environment by letter rather than number (e.g. section 6)

%\usepackage{times} % Uncomment to use the Times New Roman font

%----------------------------------------------------------------------------------------
%	DOCUMENT INFORMATION
%----------------------------------------------------------------------------------------

\title{Report for Archlab \\ Pipelined Processor} % Title

\author{Li Shengrui \\ \textsc{Lucius} \\2017012066} % Author name

\date{\today} % Date for the report

\begin{document}

\maketitle % Insert the title, author and date


%----------------------------------------------------------------------------------------
%	SECTION 1
%----------------------------------------------------------------------------------------
\part{A}
\section{Description}
\subsection{sum.ys}
\begin{lstlisting}
    # from lucius 2017012066
	.pos 0
init:
	irmovl Stack, %esp
	irmovl Stack, %ebp
	call Main
	halt

	.align 4
list:
ele1:
	.long 0x00a
	.long ele2
ele2:
	.long 0x0b0
	.long ele3
ele3:
	.long 0xc00
	.long 0

Main:
	pushl %ebp
	rrmovl %esp,%ebp
	irmovl ele1,%eax
	pushl %eax
	call sum_list
	rrmovl %ebp, %esp
	popl %ebp
	ret

sum_list:
	pushl %ebp
	rrmovl %esp, %ebp
	irmovl $0, %eax
	mrmovl 8(%ebp), %ecx
	addl %eax, %ecx
	je return
loop:
	mrmovl (%ecx), %edx
	addl %edx, %eax
	mrmovl 4(%ecx), %ecx
	irmovl $0, %esi
	addl %esi, %ecx
	jne loop
return:
	rrmovl %ebp, %esp
	popl %ebp
	ret

	.pos 0x500
Stack:
\end{lstlisting}

First, establish the initialization. Secondly, establish the list structure with sizes of 4. And with these foundation, do step by step as the Original C code indicates. It includes a function called sum\_list, which contain a loop structure.

\subsection{rsum.ys}
\begin{lstlisting}
    # from lucius 2017012066
	.pos 0
init:
	irmovl Stack, %esp
	irmovl Stack, %ebp
	call Main
	halt

	.align 4
ele1:
	.long 0x00a
	.long ele2
ele2:
	.long 0x0b0
	.long ele3
ele3:
	.long 0xc00
	.long 0

Main:
	pushl %ebp
	rrmovl %esp, %ebp
	irmovl ele1,%eax
        pushl %eax
        call rsum_list
        rrmovl %ebp, %esp
        popl %ebp
        ret

rsum_list:
	pushl %ebp
	rrmovl %esp, %ebp
	irmovl $0, %eax
	mrmovl 8(%ebp), %ecx
	addl %eax, %ecx
	je finish1

	mrmovl (%ecx), %esi
	mrmovl 4(%ecx),%eax
	pushl %esi
	pushl %eax
	call rsum_list
	popl %edi # not important
	popl %esi
	addl %esi, %eax	
	jmp finish2:

finish1:
	irmovl $0, %eax
finish2:
	rrmovl %ebp, %esp
        popl %ebp
        ret

	.pos 0x500
Stack:
 \end{lstlisting}
Very similar to the first one, This ys document establish the initialization and list structure the same way.
The difference is that this one doesn't have a loop structure. Instead, it contains a recursion structure in the middle.
before call the recursed function, it will push an argument above the stack and an callee-saved value above. 

\subsection{copy.ys}
\begin{lstlisting}
    # from lucius 2017012066
	.pos 0
init:
	irmovl Stack, %esp
	irmovl Stack, %ebp
	call Main
	halt

	.align 4
# Source block
src:
	.long 0x00a
	.long 0x0b0
	.long 0xc00

# Desitination block
dest:
	.long 0x111
	.long 0x222
	.long 0x333

Main:
	pushl %ebp
	rrmovl %esp, %ebp
	irmovl src, %esi
	pushl %esi
	irmovl dest, %esi
	pushl %esi
	irmovl $3, %esi
	pushl %esi
	call copy_block
	rrmovl %ebp, %esp
        popl %ebp
        ret

copy_block:
	pushl %ebp
        rrmovl %esp, %ebp
	mrmovl 16(%ebp), %edx
	mrmovl 12(%ebp), %ebx
	irmovl $0, %eax
	mrmovl 8(%ebp), %esi
	irmovl $4, %edi
loop:
	irmovl $0, %ecx	
	subl %ecx, %esi
	jle return
	mrmovl (%edx), %ecx
	addl %edi, %edx
	rmmovl %ecx, (%ebx)
	addl %edi, %ebx
	xorl %ecx, %eax
	irmovl $1, %ecx
	subl %ecx, %esi
	jmp loop

return:
	rrmovl %esp, %ebp
	popl %ebp
	ret

	.pos 0x500
Stack:
\end{lstlisting}
The third task is to put copies a block of words from one part of memory to another. To implement this, I move a word to a register and then move the register
to the right position. Meanwhile, it also demand us to calculate the checksum of all the words copied. And this demand only require the xorl operation in every step. 


\section{Difficulties}
For the sum.ys part, the only difficulty is initialization. But I referred to the example in the textbook, and everything went well.

And for the rsum.ys part, the difficulty is to use the callee-saved register to save $value$ in every recursion. Because $value$ is pushed above the argument $result$, I should pop twice to get $value$.

At last, for the last part copy.ys. There is many arguments to use, so I should carefully choose my register so as not to use the same register simultaneously. 

\section{What I Learned}
 In this part, I learned the basic rules to write y86 instructions, including initialization, returning and choose the right position of stuck. 
 And I also learned how to translate codes from C version to y86 version, especially in the process of calling a function. When calling a function, I should 
 push the callee-saved value and the argument. It is more complex than C language. 
%----------------------------------------------------------------------------------------
%	SECTION 2
%----------------------------------------------------------------------------------------
\part{B}
\section{Description}
\subsection{IADDL}
Description of iaddl is belowed.
\begin{lstlisting}
fetch:
    icode:ifun<-M1[PC]
    rA:rB<-M1[PC+1]
    valC<-M4[PC+2]
    valP<-PC+6	
decode:
    valB<-R[rB]
execute:
    valE<-valC+valB
    set CC
memory:

write back:
    R[rB]<-valE
PC update:
    PC<-valP	
\end{lstlisting}
This operation require a register, and a value C. PC is incremented by 6.
And this instruction is composed of irmovel and addl, it's necessary to set the condition code.


\subsection{LEAVEL}
Description of leavel is belowed.
\begin{lstlisting}
fetch:
    icode:ifun<-M1[PC]
    valP<-PC+1
decode:
    valA<-R[%ebp]
    valB<-R[%ebp]
execute:
    valE<-valB+4
memory:
    valM<-M4[valA]
write back:
    R[%esp]<-valE
    R[%ebp]<-valM
PC update:
    PC<-valP
\end{lstlisting}
This operation is to move \%esp to the position of \%ebp +4 and to move \%ebp to the position it pointed at.
And the operation doesn't need any other register and val C.  
Notice that we need 2 values of R[\%ebp] when decoding, one is used to calculate the position incremented by 4, while the other
is used to access memory.
\subsection{Difficulties}
For the SQE part, it is easy. I just check every step and add the two operation in it. As long as my instruction is correct and carefully check 
every steps, it goes well.

But for the PIPE part, things are little bit more difficult. The instrucion of the 6 steps is very similar, but I have to add more instrucions when taking
Pipeline Register Control into consideration. In other words, I must stall the register and inject a bubble for LEAVEL(It is lucky that IADDL don't need to stall or inject bubble)
Then I found that for LEAVEL, I can modify it the same way as IMRMOVL and IPOPL. Because they all need to read from the memory, the position to add stall and bubbles is the same. 
\subsection{What I Learnt}
In this part, I learnt two things. The first is how to write the description of the computations according to the demand. The Second is how to modify the register instruction.
Because the original operation has been written in the file and they share some similarities, it is easy to add new operation since the exsisted ones indicates a lot.
%----------------------------------------------------------------------------------------
%	SECTION 3
%----------------------------------------------------------------------------------------

\part{C}
\setcounter{section}{0}
\section{Instrcution ADDSL}
When we want to get the next nunber in an array, we have to let another register to be 4 and use addl. Obviously, it is troublesome. Therefore, I add an instruction named addsl. 
It means add number of a step of an array, which is 4.

e.g.
\begin{lstlisting}
# %ebp=100
addsl %ebp, %esp #Here %esp is used to hold the position; it won't be modified.
# %ebp=104
\end{lstlisting}
The specific description is belowed.
\begin{lstlisting}
fetch:
	icode:ifun<-M1[PC]
	rA:rB<-M1[PC+1]
    valP<-PC+2
decode:
    valB<-R[rB]
execute:
    valE<-4+valB
memory:

write back:
    R[rB]<-valE
PC update:
    PC<-valP	
\end{lstlisting}
So I modified the pipe-full.hcl. In addtion, I modified the isa.h, isa.c
and yas-grammer.lex under misc.

My modification is belowed.

In isa.h:
\begin{lstlisting}
	/* Different instruction types */
	typedef enum { I_HALT, I_NOP, I_RRMOVL, I_IRMOVL, I_RMMOVL, I_MRMOVL,
			   I_ALU, I_JMP, I_CALL, I_RET, I_PUSHL, I_POPL,
			   I_IADDL, I_LEAVE, I_POP2, I_ADDSL } itype_t;
\end{lstlisting}

In isa.c:
\begin{lstlisting}
	/*------------------------My change------------------------------*/
	{"addsl",  HPACK(I_ADDSL, F_NONE), 2, R_ARG, 1, 1, R_ARG, 1, 0},
	/*---------------------------------------------------------------*/

	/*------------------------My change------------------------------*/
	need_regids =
	(hi0 == I_RRMOVL || hi0 == I_ALU || hi0 == I_PUSHL ||
	 hi0 == I_POPL || hi0 == I_IRMOVL || hi0 == I_RMMOVL ||
	 hi0 == I_MRMOVL || hi0 == I_IADDL || hi0==I_ADDSL);
	 /*---------------------------------------------------------------*/

	/*------------------------My change------------------------------*/
    case I_ADDSL:
	if (!ok1) {
	    if (error_file)
		fprintf(error_file,
			"PC = 0x%x, Invalid instruction address\n", s->pc);
	    return STAT_ADR;
	}
	if (!okc) {
	    if (error_file)
		fprintf(error_file,
			"PC = 0x%x, Invalid instruction address",
			s->pc);
	    return STAT_INS;
	}
	if (!reg_valid(lo1)) {
	    if (error_file)
		fprintf(error_file,
			"PC = 0x%x, Invalid register ID 0x%.1x\n",
			s->pc, lo1);
	    return STAT_INS;
	}
	argB = get_reg_val(s->r, hi1);
	set_reg_val(s->r, hi1, argB+4);
	s->pc = ftpc;
	break;
/*---------------------------------------------------------------*/
\end{lstlisting}

In yas-grammer.lex:
\begin{lstlisting}
	Instr         rrmovl|cmovle|cmovl|cmove|cmovne|cmovge|cmovg|rmmovl|mrmovl|irmovl|addl|subl|andl|xorl|jmp|jle|jl|je|jne|jge|jg|call|ret|pushl|popl|"."byte|"."word|"."long|"."pos|"."align|halt|nop|iaddl|leave|addsl
\end{lstlisting}

And I wrote a .ys document called test1.ys, which is:
\begin{lstlisting}
	# from lucius 2017012066
	.pos 0
init:
	irmovl Stack, %esp
	irmovl Stack, %ebp
	call Main
	halt

Main:
	pushl %ebp
	rrmovl %esp,%ebp
	irmovl 0x200, %eax
	addsl %eax, %esp    # here %esp is used to hold the posotion
	rrmovl %ebp, %esp
	popl %ebp
	ret

.pos 0x500
Stack:
\end{lstlisting}

The result is exactly what I expected:
\begin{lstlisting}
	Stopped in 12 steps at PC = 0x11.  Status 'HLT', CC Z=1 S=0 O=0
	Changes to registers:
	%eax:	0x00000000	0x00000204 # Here, %eax is incremented by 4!
	%ebx:	0x00000000	0x00000100
	%esp:	0x00000000	0x00000500
	%ebp:	0x00000000	0x00000500
	
	Changes to memory:
	0x04f8:	0x00000000	0x00000500
	0x04fc:	0x00000000	0x00000011
	
\end{lstlisting}




\end{document}